(code/code10.ijm)

\begin{itemize}
\item Line 3: \ilcom{nSlices} is a macro function that returns the number of slices in the active stack. 

\item Line 4: Sets measurement parameters, from the menu would be \ilcom{[Analyze > Set measurements\ldots]}. In this case "mean min integrated" is added as part of the second argument. ``mean'' is the mean intensity, ``min'' is the minimum intensity and ``integrated'' is integrated density (total intensity). These keys for measured parameters could be known by using the command recorder. 
You do not have to care for now about the "redirect" argument. ``decimal'' is the number of digits to 
the right of the decimal point in real numbers displayed in the results table. 

\item Line 5: clears the results table. 

\item Line 6 to 9 is the loop. Loop starts from count i=0, and ends at i=frame-1. \ilcom{i++} is another way of writing \ilcom{i = i + 1}, so the increment is 1.  

\item Line 7: calculates the current frame number. 

\item Line 8: \ilcom{setSlice} function sets the frame according to the frame number calculated in line 6. 

\item Line 9:  actual measurement is done. 
Result will be recorded in the memory and will be displayed in the Results table window. 
\end{itemize}

Open an example stack \textbf{1703-2(3s-20s).stk}
\footnote{Some of you may realize that you used this sequence 
in the Image Processing / Analysis Course for learning 
stack measurements using Z-profiler \ilcom{[Image > Stacks > Plot Z-Profile]}. Now, you can program similar 
device in macro. 
Good thing about the custom program
is that you will be able to modify the program further to add more functions.
For example, You could measure the time course of standard deviation of
intensity within the selected ROI.}. This is a short sequence of FRAP analysis,
so the edge of the one of the cells is bleached and then fluorescence signal at that bleached position recovers by time. 
Select the frapped region by ROI tool (such as in the figure below). 
Execute the macro. Results will be printed in the Results window (see the table in the figure right: this table is showing only "Mean" column as only ``Mean Intensity'' was selected in the measurement option).